\documentclass{article}
% Comment the following line to NOT allow the usage of umlauts
\usepackage[utf8]{inputenc}
% Uncomment the following line to allow the usage of graphics (.png, .jpg)
%\usepackage{graphicx}

% Start the document
\begin{document}

% Create a new 1st level heading
\section{""FUTURE OF HEALTH CARE""}
Telehealth has made it possible for patients to receive care without an in-
person office visit. Remote patient monitoring is becoming more widely ac-
cepted. This now includes wearable technology with impressive capabilities,
from remote monitoring of vitals to remote echocardiograms. If not for the
pandemic, it would have taken the healthcare industry another decade to
reach where it is today.
Achieving universal health coverage (UHC) requires health financing systems
that provide prepaid pooled resources for key health services without placing
undue financial stress on households. Understanding current and future trajec-
tories of health financing is vital for progress towards UHC. We used historical
health financing data for 188 countries from 1995 to 2015 to estimate future
scenarios of health spending and pooled health spending through to 2040.


In the third century B.C.E., the ancient Greek historian Polybius maintained
that we should not examine societies in isolation. He insisted on looking at the
world as a whole , which would help us understand countries in context and
compare them to one another. As a result, health systems in the modern era
should be addressed on a global scale, yet they rarely are.



Achieving universal health coverage (UHC) requires health.
What are the major broad-scale trends affecting health-care systems? We identi-
fied five recurring issues after conducting a thorough review of the health policy
and systems literature: sustainable health systems, the genomics revolution,
emerging technologies, global demographic dynamics, and new care models. We
give a quick rundown of each.

\subsection{Nanotechnology}
 nanotechnology have the potential to make
a very significant impact on society. In general it may be assumed that
the application of nanotechnology will be very beneficial to individuals and
organisations. These include materials in the form of very thin films used
in catalysis and electronics, two-dimensional nanotubes and nanowires for
optical and magnetic systems, and as nanoparticles used in cosmetics, phar-
maceuticals and coatings. The industrial sectors most readily embracing
nanotechnology are the information and communications sector, including
electronic and optoelectronic fields, food technology, energy technology and
the medical products sector, including many different facets of pharmaceu-
ticals and drug delivery systems, diagnostics and medical technology, where
the terms nanomedicine and bionanotechnology are already commonplace.



Nanotechnology products may also offer novel challengies for the reduction
of environmental pollution.
Although in the natural world there
are many examples of structures that exist with nanometre dimensions (here-
after referred to as the nanoscale), including essential molecules within the
human body and components of foods, and although many technologies have
incidentally involved nanoscale structures for many years, it has only been
in the last quarter of a century that it has been possible to actively and in-
tentionally modify molecules and structures within this size range. It is this
control at the nanometre scale that distinguishes nanotechnology from other
areas of technoloNanotechnology involve the ability to see and to control individual atoms
and molecules. 





Everything on Earth is made up of atoms—the food we eat,
the clothes we wear, the buildings and houses we live in, and our own bodies.
The ideas and concepts behind nanotechnology started with a talk enti-
tled “There’s Plenty of Room at the Bottom” by physicist Richard Feynman
at an American Physical Society meeting at the California Institute of Tech-
nology (CalTech) on December 29, 1959, long before the term nanotechnol-
ogy was used. 


In his talk, Feynman described a process in which scientists
would be able to manipulate and control individual atoms and molecules.
Nanotechnology is the term given to those areas of science and engineering
where phenomena that take place at dimensions in the nanometre scale are
utilised in the design, characterisation, production and application of mate-
rials, structures, devices and systems

\subsection{telehealth}

The Health Resources Services Administration defines telehealth as the
use of electronic information and telecommunications technologies to support
long-distance clinical health care, patient and professional health-related edu-
cation, public health and health administration.




 Technologies include video-
conferencing, the internet, store-and-forward imaging, streaming media, and
terrestrial and wireless communications.Telemedicine, also referred to as telehealth or e-medicine, is the remote
delivery of healthcare services, including exams and consultations, over the
telecommunications infrastructure. Telemedicine allows healthcare providers
to evaluate, diagnose and treat patients without the need for an in-person
visit. Patients can communicate with physicians from their homes by using
their own personal technology or by visiting a dedicated telehealth kiosk.





Telehealth is the use of digital information and communication technologies,
such as computers and mobile devices, to access health care services remotely
and manage your health care. These may be technologies you use from home
or that your doctor uses to improve or support health care services.After collecting passive data, IoT healthcare devices
would send this critical information to the cloud so that
doctors can act upon it. Thus, IoT-based healthcare ser-
vices not only improve a patient’s health and help in crit-
ical situations but also the productivity of health employ-
ees and healthcare organizations’ workflows




Telehealth is different from telemedicine because it refers to a broader
scope of remote healthcare services than telemedicine. While telemedicine
refers specifically to remote clinical services, telehealth can refer to remote
non-clinical services, such as provider training, administrative meetings, and
continuing medical education, in addition to clinical services. According to
WHO, telemedicine is the deliveryof healthcare services by using information
andcommunication technologies for the exchange of information for diagnosis,
treatment and prevention of disease. 


“Telemedicine” is often still used when
referring to traditional clinical diagnosis and monitoring that is delivered by
technology. “Telehealth” is now more commonly used as it describes the
wide range of diagnosis and management, education, and other related fields
of health care.
2

% Uncomment the following two lines if you want to have a bibliography
%\bibliographystyle{alpha}
%\bibliography{document}

\end{document}
